%% main file for the C++ standard.
%%

%%--------------------------------------------------
%% basics
\documentclass[ebook,10pt,oneside,openany,final]{memoir}

\newcommand{\docno}{P0539R5}
\newcommand{\prevdocno}{P0539R4}
\newcommand{\reldate}{\today}

\usepackage[american]
           {babel}        % needed for iso dates
\usepackage[iso,american]
           {isodate}      % use iso format for dates
\usepackage[final]
           {listings}     % code listings
\usepackage{longtable}    % auto-breaking tables
\usepackage{ltcaption}    % fix captions for long tables
\usepackage{booktabs}     % fancy tables
\usepackage{relsize}      % provide relative font size changes
\usepackage{underscore}   % remove special status of '_' in ordinary text
\usepackage{verbatim}     % improved verbatim environment
\usepackage{parskip}      % handle non-indented paragraphs "properly"
\usepackage{array}        % new column definitions for tables
\usepackage[normalem]{ulem}
\usepackage{color}        % define colors for strikeouts and underlines
\usepackage{amsmath}      % additional math symbols
\usepackage{mathrsfs}     % mathscr font
\usepackage{microtype}
\usepackage{multicol}
\usepackage{xspace}
\usepackage{fixme}
\usepackage{lmodern}
\usepackage[T1]{fontenc}
\usepackage[pdftex, final]{graphicx}
\usepackage[pdftex,
            pdftitle={A Proposal to add wide_int Template Class},
            pdfsubject={A Proposal to add wide_int Template Class},
            pdfcreator={Antony Polukhin},
            bookmarks=true,
            bookmarksnumbered=true,
            pdfpagelabels=true,
            pdfpagemode=UseOutlines,
            pdfstartview=FitH,
            linktocpage=true,
            colorlinks=true,
            linkcolor=blue,
            plainpages=false
           ]{hyperref}
\usepackage{memhfixc}     % fix interactions between hyperref and memoir
\usepackage{xstring}

\input{layout}
\input{styles}
\input{macros}
\input{tables}


%%--------------------------------------------------
%% fix interaction between hyperref and other
%% commands
\pdfstringdefDisableCommands{\def\smaller#1{#1}}
\pdfstringdefDisableCommands{\def\textbf#1{#1}}
\pdfstringdefDisableCommands{\def\raisebox#1{}}
\pdfstringdefDisableCommands{\def\hspace#1{}}

%%--------------------------------------------------
%% add special hyphenation rules
\hyphenation{tem-plate ex-am-ple in-put-it-er-a-tor name-space name-spaces}

%%--------------------------------------------------
%% turn off all ligatures inside \texttt
\DisableLigatures{encoding = T1, family = tt*}

%%--------------------------------------------------
%% make overfull hboxes visible
%\setlength\overfullrule{5pt}

\begin{document}
\chapterstyle{cppstd}
\pagestyle{cpppage}

%%--------------------------------------------------
%% configuration

%% Title
\newcommand{\doctitle}{A Proposal to add wide_int Template Class}



%%--------------------------------------------------
%% front matter
\frontmatter

\begingroup
\def\hd{\begin{tabular}{ll}
          \textbf{Document Number:} & {\larger\docno}             \\
          \textbf{Date:}            & \reldate                    \\
          \textbf{Audience:}        & SG6, LEWGI, LEWG            \\
          \textbf{Revises:}         & \prevdocno                  \\
          \textbf{Reply to:}        & Igor Klevanets              \\
                                    & cerevra@yandex.ru, cerevra@yandex-team.ru \\
                                    & Antony Polukhin \\
                                    & antoshkka@gmail.com, antoshkka@gmail.com \\
          \end{tabular}
}
\newlength{\hdwidth}
\settowidth{\hdwidth}{\hd}
\hfill\begin{minipage}{\hdwidth}\hd\end{minipage}
\endgroup

\vspace{2.5cm}
\begin{center}
\textbf{\Large\doctitle}
\end{center}

%%--------------------------------------------------
%%--------------------------------------------------
%%--------------------------------------------------



\section{I. Introduction and Motivation}

Current standard provides signed and unsigned \tcode{int8_t}, \tcode{int16_t}, \tcode{int32_t}, \tcode{int64_t}. It is usually enough for every day tasks, but sometimes appears a need in big numbers: for cryptography, IPv6, very big counters etc. Non-standard type \tcode{__int128} which is provided by GCC and Clang illuminates this need. But there is no cross-platform solution and no way to satisfy future needs in even more big numbers.

This is an attempt to solve the problem in a generic way on a library level and provide wording for \href{https://wg21.link/P0104R0}{P0104R0: Multi-Word Integer Operations and Types}.

A proof of concept implementation available at: \href{https://github.com/cerevra/int/tree/master/v3}{https://github.com/cerevra/int/tree/master/v3}. 



\section{II. Changelog}

Differences with \href{https://wg21.link/P0539R4}{P0539R4}:
\begin{itemize}
\item Sync with P1889R1:
    \begin{itemize}
    \item Use \LaTeX{} for wording and paper
    \item Use spaceship operator
    \end{itemize}
\end{itemize}

Differences with \href{https://wg21.link/P0539R3}{P0539R3}:
\begin{itemize}
\item More paragraphs in "III. Design and paper limitations"
\item Added requirements on \tcode{wide_integer} alignment and layout
\item Dropped the \tcode{explicit} from \tcode{operator bool()}
\item Permitted implementations to add overloads (mostly for adding explicit overloads that do not generate warnings)
\end{itemize}

Differences with \href{https://wg21.link/P0539R2}{P0539R2}:
\begin{itemize}
\item \textbf{"Bits" won in the discussion "Words vs Bytes vs Bits" in Albuquerque}
\item Changed "MachineWords" to "Bits"
\item Interoperability with other types moved to a separate paper
\item Removed signedness scoped enum
\end{itemize}

Differences with \href{https://wg21.link/P0539R1}{P0539R1}:
\begin{itemize}
\item Added a discussion on "Words vs Bytes vs Bits"
\end{itemize}

Differences with \href{https://wg21.link/P0539R0}{P0539R0}:
\begin{itemize}
\item Reworked the proposal for simpler integration with other Numerics proposals:
    \begin{itemize}
       \item Added an interoperability section [numeric.interop]
       \item Arithmetic and Integral concepts were moved to [numeric.requirements] as they seem widely useful
       \item Binary non-member operations that accept Arithmetic or Integral parameter were changed to accept two Arithmetic or Integral parameters respectively and moved to [numeric.interop]
       \item \tcode{int128_t} and other aliases now depend on \href{https://wg21.link/P0102R0}{P0102R0}
    \end{itemize}
\item Renamed wide_int to \tcode{wide_integer}
\item \tcode{wide_integer} now uses machine words count as a template parameter, not bits count
\item Removed not allowed type traits specializations
\item Added \tcode{to_chars} and \tcode{from_chars}
\end{itemize}



\section{III. Design and paper limitations}
\tcode{wide_integer} is designed to be as close as possible to built-in integral types:
\begin{itemize}
\item it does not allocate memory
\item it is a standard layout type
\item it is trivially copyable
\item it is trivially desctructible
\item it is constexpr usable
\end{itemize}

\tcode{wide_integer} is \textbf{not} a metafunction. Such metafunctions are discussed in \href{https://wg21.link/P0102}{P0102}.

\tcode{wide_integer} does \textbf{not} add \tcode{noexcept} to operations that have UB for built-in types. Operations that have UB traditionally are not marked with \tcode{noexcept} by LWG. Attempt to change that behavior may be done in separate paper.

In this proposal we concentrate on the \tcode{wide_integer} class that uses machine words under the cover. Such implementations allow you to get best performance and leave behind some design questions, like "Is \tcode{(sizeof(wide_integer<X>) == X / CHAR_BIT)} or not?".

However, we do not wish to shut the door close for extending the abilities of the \tcode{wide_integer} class. Some users may wish to see \tcode{unit40_t}, or \tcode{unit48_t}.

We insist on \textbf{interface that allows specifying integers not representable by machine words count}. Such extensions of functionality may be discussed in separate papers.

\textbf{\tcode{wide_integer} mimics the behavior of \tcode{int}}. Because of that \tcode{wide_int<128> * wide_int<128>} results in \tcode{wide_int<128>}, not \tcode{wide_int<256>}. There are separate proposals for integers that are elastic (\href{https://wg21.link/P0828}{P0828}) or safe (\href{https://wg21.link/P0228R0}{P0228R0}).

Non template aliases for integers of particular width (for example \tcode{int128_t}) are handled in separate paper P0102R0. There was an implicit request from LEWG to remove those aliases from this paper, as they were rising a lot of questions on some platforms.

Interoperability with other arithmetic types was moved to a separate paper \href{https://wg21.link/P0880}{P0880}.

We double checked that constexpr on default constructor does not require zero initialization in non constexpr contexts and still allows zero initialization if explicitly asked:
\begin{codeblock}
int main() {
     //constexpr wide_integer<128, unsigned> wi;   // Not initialized in constexpr context - compile time error
     wide_integer<128, unsigned> wi_no_init;       // Not initialized - OK
     constexpr wide_integer<128, unsigned> wi{};   // Zero initialized - OK
}
\end{codeblock}

Current revision of the paper mimics behavior of int even in cases that are considered dangerous. It is possible to make \tcode{wide_integer} type more safe by making explicit:
\begin{itemize}
\item Conversion from any signed to any unsigned type.
\item Narrowing conversions.
\item Conversion to and from floating point types.
\item Conversion to \tcode{bool}.
\end{itemize}

Such change will break the compatibility of \tcode{wide_integer} and \tcode{int}. Consider the case, when you have some template function \tcode{foo(Arithmetic a)} that works with arithmetic types. Function is huge and it was developed a long time ago. If interface of \tcode{wide_integer} is same as the interface of int then \tcode{foo()} could work with it. But if we make some of the conversions explicit, then \tcode{foo()} function must be adjusted.

Also note that adding explicit would affect only \tcode{wide_integer}s, while there could be a better solution for all the integral types. For example all the integrals could be fixed by some \tcode{safe_integer<Integral>} class, that is very explicit and does additional checks on demand. In that case adding restrictions into \tcode{wide_integer} would just break the interface compatibility with \tcode{int} without big benefit.


\section{IV. Acknowledgements}

Many thanks to Alex Strelnikov and John McFarlane for sharing useful ideas and thoughts.



\section{V. Feature-testing macro}

For the purposes of SG10 we recommend the feature-testing macro name \tcode{__cpp_lib_wide_integer}.



\section{VI. Proposed wording}

%%--------------------------------------------------
%%--------------------------------------------------
%%--------------------------------------------------

%%--------------------------------------------------
%% Wording

\mainmatter
\setglobalstyles
\include{wideinteger}


%%--------------------------------------------------
%% End of document
\end{document}
